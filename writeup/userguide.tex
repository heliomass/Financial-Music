\section{Appendix E: User Guide}

Welcome to the user guide. Here you will learn how to operate Financial Music.

\subsection*{System Requirements}

Before beginning, you will need to ensure you have Java 1.6 and Jython 2.2.1 installed. Java can be obtained from \url{www.sun.com} and Jython can be obtained from \url{www.jython.org}

\subsection*{File Formats}

Financial Music expects the account information to be in CSV (Comma Seperated Value) files. These can be opened or generated with a spreadsheet package such as Microsoft Excel.

\subsection*{Compiling Financial Music}

To prepare Financial Music for operation, you will need to open a command line console. Once this is done, navigate to the directory containing Financial music and type: \texttt{java *.java}. This will perform the compilation.

\subsection*{Music Generation}

Before generating the music, you will need to make sure that you have two subsequent years' balance sheets in two separate CSV files.\\

To generate music for the Signal Mapping implementation, type the following at the command line: \texttt{jython Mapping.py year1 year2}. Likewise, to generate music for the L-System Music Generation implementation, type: \texttt{jython LSS.py year1 year2}. In both instances, replace \texttt{year1} and \texttt{year2} with the appropriate file names.

\subsection*{Playing the Music}

Once the music is generated, you can hear it by typing the following at the command line: \texttt{java MusicReader}.

\subsection*{Changing Program Settings}

To alter the settings used to generate the music, open your favourite text editor and choose `open file'. Navigate to the Financial Music directory, and open \texttt{settings.py}. Within, you will find instructions on how to change these settings.